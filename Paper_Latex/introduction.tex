\section{Introduction}
Caching is one of the oldest and most fundamental metaphor in modern computing. The main problem of cache management is to design a replacement policy that maximizes the hit rate measured over a very long trace subject to the important practical constraints of minimizing the computational and space overhead involved in implementing the policy. Generally, CPU cache, its replacement policy implementation is limited to be simple due to the hardware-bounded constraints. e.g. logic-circuit design, processor support, cache-associativity, etc. However, for disk cache, such constraints are normally relaxed compared to CPU cache. i.e. there exists a chance to improve the hit ratio by developing a good cache replacement policy. From 1960s to 2000s, many research paper has been published to solve this problem, But a well-known, folklore algorithm, called LRU(Least Recently Used)~\cite{lru} showed a better performance by either hit ratio or tuning fashion. Even though there were some policies~\cite{lrfu},~\cite{mq} that show better hit ratio than LRU, they have critical weak points that they can outperform than LRU when some tunable paramter is fixed with some value. The existence of tuning-parameter seized them to overcome LRU policy due to the diversity of workload and computing environments since the value of paramter in each policy is highly depended on the workloads. e.g. database queries. 
To cope with this tuning-constraints, a novel cache replacement policy for disk cache called ARC(Adaptive Replacement Cache)~\cite{arc} has been published in 2003. Unlike other policies, ARC has no tunable parameters since it adaptively changes the parameters itself during its execution by its learning rate. They showed significant improvement compared to LRU with respect to both hit ratio and tuning-constraints. Accordingly, ARC and its variants have been widely used in storage system nowadays. e.g. IBM storage controller DS6000, ZFS file system~\cite{zfs}, PostgreSQL~\cite{postgresql}, VMware's vSAN~\cite{vsan}. 

Although the author of ARC~\cite{arc} claims that it is "self-tunable", However, in our point of view, there still exists some tunable parameters. First, they use initial value of $p$ as 0(which is the size of LRU portion in cache), but there's no guarantee that it is optimal for all workloads. Second, when they encounter phantom cache hit(i.e. actually miss, but exists in ghost cache), they use learning rate to increase/decrease size of LRU-LFU~\cite{lru} portion respectively or vice versa. However, there is no proof that the value of learning rate is reasonable for all workloads. Lastly, they limit the maximum size of $p$ as the cache size in ARC. However, this might loose full utilization of using ghost cache when they encountered a LRU-intensive workload. To investigate these conjectures, We first precisely speculate this observation. Second, we evaluate the hit-ratio of our ``After-ARC" variation compared to original ARC on real-life benchmark disk I/O traces. Since various second-storages e.g. Flash-memory, SSD(Solid State Drive) have emerged today, it is not reasonable for only measuring the hit ratio of the disk cache. However, this consideration is beyond the scope of this paper. We only aim at comparing our proposed variation of ARC is valid compared to original ARC. In terms of that, our evaluation is still valid. Additionally, we verify that ARC is still effective on the workloads that we use these days because it has been 16 years passed since ARC has been published and there have been a lot of changes on I/O patterns in several workloads. Our experimental results say that proposed speculation can be used to improve the hit ratio of the cache and ARC still derives efficient hit ratio for the disk cache.

This paper is organized as follows. In Section 2, we survey related works. Then, Section 3 briefly explains the idea of adapative cache replacement policy which we want to precisely  evaluate in terms of "self-tunable". In Section 4, we first explain our target traces i.e. real-life benchmarks that is widely used in research-area. And we also briefly describe about our simulation environment, and evaluates the experimental results. In Section 6, we discuss about the correctness of our consideration based on some interesting experimental results. Finally, Section 7 concludes the paper.
