\section{Discussion}

About practicality, we cannot say it is practical since we does not consider the trend of the second-storages. e.g. Flash-Memory, SSD. Espeically for flash-memory, it involves the write amplification and wear-leveling due to P/E cycle which affects the life-cycle of flash memory. We only measured the hit ratio of the disk traces, However, we expect that there is some amplifications for disk operations due to garbage collection or wear-leveling. Despite of that, we investigate the validity of ARC in terms of 3 tunable parameters and find that 1) EXP is quite good-option, 2) ARC is still useful replacement policy in current-days workloads. If interested in second storage properties, refer these following research papers~\cite{ssd1},~\cite{ssd2},~\cite{cflru},~\cite{flash}.\\
About experiment, we collected about 30 traces. e.g. traces from SYSTOR17 Lee et al traces~\cite{systor17}, and other traces from ~\cite{microsoft} e.g. TPC-C, TPC-E, LM-TFE, WBS, DTRS, and etc. we spent about 4 days for parsing the disk I/O patterns of each trace. There was a lack of time for simulating all 30 traces within short days. So we only showed the simulation results for 8 traces. we spent about 8 hours for simulating those 8 traces. It would be more nice to measure the other 22 traces with the same configurations in Section V. Also, we fixed the size of disk cache as 256MB, but the size of disk cache in currently released servers varies from 128MB to even 2GB. We think that it would be also useful to simulate those traces with a much higher disk cache size. All sources used in this paper are freely available on our GitHub repository~\cite{opensource}. Note that even each parsed data was so big to upload at web-site, we omitted the actual/parsed traces. However, the code for parsing disk traces are available on web-site~\cite{opensource} and all actual traces are freely available on SNIA/IOTTA~\cite{snia}. For researchers who are interested in this paper, we anticipate that anyone can easily use our open-source code. Hopefully, it would be nice that our source could be used for the basis of the further research or implementation for ARC or variants of ARC.

